\documentclass[a4paper, titlepage]{article}

\usepackage[ngerman]{babel}
\usepackage[utf8]{inputenc}
\usepackage[T1]{fontenc}
\usepackage{lmodern}
\usepackage[style=numeric,backend=biber]{biblatex}
\usepackage{csquotes}
\usepackage{hyperref}
\usepackage{graphicx}

\graphicspath{ {./images/}}
\addbibresource{references.bib}

\title{Echoes}
\author{Damien Flury, Tim Hess}
\date{30. November 2018}

\begin{document}
    \maketitle
    \tableofcontents
    \newpage

    \section{Einführung}
    \subsection{Auftrag}
    Unser Auftrag besteht darin, eine Applikation mit Datenbankanbindung zu entwickeln. Technologien
    können frei gewählt werden.
    \subsection{Idee}
    Unsere Idee ist die Entwicklung einer Webapplikation für die Planung von Hausaufgaben
    und Prüfungen. 
    \subsection{Planung}
    Die Webapplikation soll die Verwaltung verschiedener Klassen mit Schülern ermöglichen.
    Man soll einen Account erstellen, Hausaufgaben und Prüfungen zu einer Klasse hinzufügen
    und diese auch als erledigt markieren können.
    \section{Technologien}
    Für die Webapplikation verwenden wir verschiedene Frameworks. Serverseitig bieten wir
    eine REST API mit ASP.NET Core \cite{Dotnet} an, Clientseitig konsumieren wir diese API mit dem Single
    Page Application (SPA) Framework Angular \cite{Angular}.
    \subsection{Backend}
    \begin{itemize}
        \item ASP.NET Core 2.1
        \item C\# 7.3
        \item Microsoft SQL Server
    \end{itemize}
    \subsection{Frontend}
    \begin{itemize}
    \item Angular 7
    \end{itemize}
    \begin{figure}
        \includegraphics[width=\textwidth]{angular}
        \caption{Angular}
    \end{figure}
    \begin{figure}
        \includegraphics[width=\textwidth]{csharp}
        \caption{C\#}
    \end{figure}

    \newpage
    \printbibliography
\end{document}
